\documentclass[11pt,a4paper]{report}

\usepackage[utf8]{inputenc}
\usepackage{titling}
\usepackage[german]{babel}
\usepackage[T1]{fontenc}
\usepackage{amsmath}
\usepackage{amsfonts}
\usepackage{amssymb}
\usepackage[left=3cm,right=2cm,top=2.5cm,bottom=2cm]{geometry}
\usepackage{graphicx}
\usepackage{fancyhdr}
\pagestyle{fancy}

\lhead{Leon Rudolph}
\chead{ak18b}
\rhead{17.12.2018}

\lfoot{}
\cfoot{\thepage}
\rfoot{}

\renewcommand{\headrulewidth}{0.4pt}
\renewcommand{\footrulewidth}{0.4pt}

\begin{document}
\begin{titlepage}

\pretitle{%
\vskip -3em
\begin{figure}[h]
\begin{center}
\includegraphics[scale=0.55]{Logo.png}
\end{center}
\end{figure}
\begin{center}
\vskip -2em
\large{Wintersemester 2018/19\\Softwaretechnikpraktikum} \vskip 9em
\rule{5in}{0.4pt}\par \vskip 0.5em
}
\posttitle{\par\rule{5in}{0.4pt} \vskip 4em
\Large Gruppe: ak18b \vskip 1.5em
\normalsize Betreuer: Benjamin Lucas Friedland, Michael Fritz\vskip 1em
\normalsize Gruppenmitglieder: Alexander Zwisler, Leon Kamuf, Leon Rudolph, Maurice Eisenblätter, Maximilian Gläfcke, Robin Seidel, Sina Opitz, Steve Woywod
\end{center}}

\title{\textbf{\Huge App zur Inventarisierung von Unternehmenswerten}\vskip 0.5em \huge Projektplan}
\date{}
\maketitle
\end{titlepage}
\setcounter{secnumdepth}{4}
\setcounter{tocdepth}{4}
\tableofcontents
\thispagestyle{empty}
\newpage
\setcounter{page}{1}
\renewcommand\thesection{\arabic{section}}

\section{Projektplan}

\subsection{Inhaltsverzeichnis}

\subsubsection{Ziele}

Muss und Kann Ziele entstammen den funktionalen und nicht-funktionalen Anforderungen, des Lastenhefts
  \begin{itemize}
  \item 9 Muss Ziele 72\% des Gesamtprojekts
  \item 7 Kann Ziele 45\% des Gesamtprojekts
  \end{itemize}



\begin{itemize}
\item[]
  \textbf{Muss Ziele}
  
  \begin{itemize}
  	\item
  		An- und Abmeldung
  	\item
  		User Self Care
	\item
  		Datenzugriff
	\item
 		Dateneingabe
	\item
  		Typeneingabe
	\item
  		Administration
	\item
  		Datenschutz und Anonymität
	\item
  		Fehlerunanfälligkeit
	\item
  		Benutzeroberfläche
  \end{itemize}

\item[]
  \textbf{Kann Ziele}
  
  \begin{itemize}
  	\item
  		Mehrsprachigkeit
	\item
  		Itemtyp Vorlagen
	\item
  		Statistiken für Itembestände
	\item
  		Benutzerhandbuch
	\item
  		Chat
	\item
  		Nutzerverwaltung
	\item
  		Browsersupport
  \end{itemize}
  
\end{itemize}

\newpage

\subsubsection{Arbeitspakete}\label{arbeitspakete}

\begin{itemize}
\item
  \textbf{Abgabe Arbeitspaket 1}
  
  \begin{itemize}
  	\item
  		An- und Abmeldung
  		\begin{itemize}
  		\leftskip=3em
  		\item[/LF0010/] Login
  		\item[/LF0020/] Logout
  		\end{itemize}
  	\item
    	User Self Care
    	\begin{itemize}
    	\leftskip=3em
    	\item[/LF0110/] E-Mail Adresse ändern
    	\item[/LF0120/] Passwort ändern
    	\end{itemize}
    \item
    	Datenzugriff
    	\begin{itemize}
    	\leftskip=3em
    	\item[/LF0210/] Anzeigen aller erfassten Items
    	\item[/LF0230/] Sortieren von Items
    	\end{itemize}
    \item
    	Dateneingabe
    	\begin{itemize}
    	\leftskip=3em
    	\item[/LF0330/] Bearbeiten von Items
    	\item[/LF0340/] Entfernen von Items
     	\item[/LF0360/] Verlinkung von Items
    	\end{itemize}
  	\item
    	Typeneingabe
    	\begin{itemize}
    	\leftskip=3em
    	\item[/LF0410/] Hinzufügen neuer Itemtypen
    	\item[/LF0420/] Bearbeiten von Itemtypen
    	\item[/LF0430/] Pflichtfelder von Itemtypen
    	\item[/LF0440/] Löschen von Itemtypen
    	\end{itemize}
  	\item
    	Administration
    	\begin{itemize}
    	\leftskip=3em
    	\item[/LF0540/] Typenverwaltung
    	\item[/LF0550/] Globale Pflichtfelder
    	\end{itemize}
  \end{itemize}

\item
  \textbf{Abgabe Arbeitspaket 2}
  
  \begin{itemize}
  	\item
  		An- und Abmeldung
  		\begin{itemize}
  		\leftskip=3em
  		\item[/LF0030/] Registrierung neuer Anwender
  		\item[/LF0040/] Wiederherstellung des Passworts
  		\end{itemize}
	\item
    	Datenzugriff
    	\begin{itemize}
    	\leftskip=3em
    	\item[/LF0220/] Suche
    	\end{itemize}
	\item
    	Dateneingabe
    	\begin{itemize}
    	\leftskip=3em
    	\item[/LF0350/] Bearbeiten von Items
    	\end{itemize}
	\item
    	Administration
    	\begin{itemize}
    	\leftskip=3em
    	\item[/LF0520/] Löschen von Profilen
    	\item[/LF0530/] Firmenverwaltung
    	\end{itemize}
  \end{itemize}
\newpage
\item
  \textbf{Abgabe Arbeitspaket 3}
  
  \begin{itemize}
  	\item
    	Dateneingabe
    	\begin{itemize}
    	\leftskip=3em
    	\item[/LF0320/] Anhang von Dokumenten
    	\end{itemize}
  	\item
  		Datenschutz und Anonymität
  		\begin{itemize}
  		\leftskip=3em
  		\item[/LN0010/] Sicherheit der Anmeldedaten
  		\end{itemize}
	\item
  		Fehlerunanfälligkeit
  		\begin{itemize}
  		\leftskip=3em
  		\item[/LN0110/] Validieren von Benutzereingaben
  		\item[/LN0120/] Datensicherheit
  		\end{itemize}
	\item
  		Benutzeroberfläche
  		\begin{itemize}
  		\leftskip=3em
  		\item[/LN0310/] Intuitive grafische Interaktion
  		\end{itemize}
	\item
  		Datenschutz und Anonymität
  		\begin{itemize}
  		\leftskip=3em
  		\item[/LN0410/] Browsersupport
  		\end{itemize}
  \end{itemize}

\item
  \textbf{Abgabe Arbeitspaket 4}
  
  \begin{itemize}
  	\item
  		Mehrsprachigkeit
  		\begin{itemize}
  		\leftskip=3em
  		\item[/LF0610/] Mehrsprachigkeit
  		\end{itemize}
	\item
  		Itemtyp Vorlagen
  		\begin{itemize}
  		\leftskip=3em
  		\item[/LF0620/] Itemtyp Vorlagen
  		\end{itemize}
	\item
  		Statistiken für Itembestände
  		\begin{itemize}
  		\leftskip=3em
  		\item[/LF0630/] Statistiken für Itembestände
  		\end{itemize}
	\item
  		Benutzerhandbuch
  		\begin{itemize}
  		\leftskip=3em
  		\item[/LF0640/] Benutzerhandbuch
  		\end{itemize}
	\item
  		Chat
  		\begin{itemize}
  		\leftskip=3em
  		\item[/LF0650/] Chat
  		\end{itemize}
  	\item
  		Nutzerverwaltung
  		\begin{itemize}
  		\leftskip=3em
  		\item[/LF0660/] Nutzerverwaltung
  		\end{itemize}
  \end{itemize}

\end{itemize}
\newpage

\subsection{Beschreibung der Arbeitspakete}

\subsubsection{Abgabe Paket 1 (37\% des Gesamtprojekts)}
Das Abgabe Paket 1 umfasst eine grobe Darstellung der Website, welches die grundlegende Bedienung für den Anwender bereitstellt.
Es wird eine Benutzeroberfläche erstellt, die Funktionen sich an-, abzumelden, zu registrieren oder sein Passwort zu ändern, sowie die Typen- und Dateneingabe hinzugefügt.
Die Typen- und Dateneingabe setzen sich mit der Thematik auseinander neue Items zum Inventar des Unternehmens hinzu zu fügen.
An- und Abmeldung soll einem Unternehmen ermöglichen, dass nur bestimmte Personen zugriff zur Website besitzen, gegebenenfalls können sich außenstehende noch registrieren um Zugriff zu erlangen.
Unser erstes Abgabe Paket macht prozentual den größten Anteil im Gesamtprojekt aus, allerdings soll es auch für den Anfang möglichst viel bieten um eine grobe Idee der Inventarisierung darzustellen.
\begin{itemize}
\itemsep1pt\parskip0pt\parsep0pt
\item
  An- und Abmeldung (8\% des Gesamtprojekts)

  \begin{itemize}
  \item
    Diese Funktion bietet die Möglichkeit, sich über Nutzerdaten anzumelden, abzumelden
    oder sich zu registrieren, um neue Nutzerdaten zu erhalten.
  \end{itemize}
  
\item
  Passwortänderung (5\% des Gesamtprojekts)

  \begin{itemize}
  \item
    Diese Funktion ermöglicht das Passwort zu ändern.
  \end{itemize}
  
\item
  Dateneingabe (8\% des Gesamtprojekts)

  \begin{itemize}
  \item
    Diese Funktion ermöglicht das Entfernen, Bearbeiten, Hinzufügen und Anhängen neuer
    Items.
  \end{itemize}
  
\item
  Typeneingabe (8\% des Gesamtprojekts)

  \begin{itemize}
  \item
    Wie bei der Dateneingabe umfasst die Typeneingabe das Hinzufügen,
    Bearbeiten und Entfernen der jeweiligen Itemtypen.
  \end{itemize}
  
\item
  Benutzeroberfläche (8\% des Gesamtprojekts)

  \begin{itemize}
  \item
    Die Funktion ermöglicht ein übersichtliches Userinterface zu erstellen.
  \end{itemize}
  
\end{itemize}

\subsubsection{Abgabe Paket 2 (26\% des Gesamtprojekts)}
Die Abgabe des zweiten Pakets umfasst das Hinzufügen neuer Funktion um die Website dem Ziel einer Iventarisierungs-App näher zu bringen.
Nachdem das erste Paket eine grobe Darstellung lieferte, stellt das Abgabe Paket 2 das Hinzufügen der Funktion der Itemvorlage, Verlinkung, Administration und dem Datenzugriff bereit.
Diese setzen sich aus den Kann und Muss Zielen zusammen und ermöglichen dem Nutzer beispielsweise, das Suchen nach bestimmten Items durch die Funktion Datenzugriff.
Prozentual stellt das zweite Abgabe Paket weniger zum Gesamtprojekt bereit als sein Vörgänger, liefert dafür allerdings ein umfassendes hinzufügen neuer Funktion für den Anwender.
\begin{itemize}
\item
  Itemvorlage (5\% des Gesamtprojekts)

  \begin{itemize}
  \item
    Bei häufiger Verwendung von Items werden diese mithilfe dieser Funktion bereits hinterlegt
    und vorgeschlagen.
  \end{itemize}
  
\item
  Verlinkung von Items (5\% des Gesamtprojekts)

  \begin{itemize}
  \item
    Die Items werden untereinander verlinkt.
  \end{itemize}
  
\item
  Datenzugriff (8\% des Gesamtprojekts)

  \begin{itemize}
  \item
    Die Items werden in einer Tabelle erfasst und können mit dieser Funktion ebenfalls
    gesucht werden.
  \end{itemize}
  
\item
  Administration (8\% des Gesamtprojekts)

  \begin{itemize}
  \item
    Die Administration umfasst das Hinzufügen von Firmen oder
    Unternehmen, sowie die Typenverwaltung als auch das Hinzufügen von
    globalen Pflichtfeldern.
  \end{itemize}
  
\end{itemize}

\subsubsection{Abgabe Paket 3 (23\% des Gesamtprojekts)}
Abgabe Paket 3 beschäftigt sich mit dem Hinzufügen von Funktion, die im Gegensatz zu Paket 2, komplexer gestaltet sind.
Die Filterung und Sortierung der Items ist zwar unter dem Punkt Kann Ziele gelistet weist allerdings auch eine hohe Fehlerangegebenenfallsfälligkeit oder Komplexität auf.
Diese Funktion bietet nur 5\% zur Fertigstellung des Gesamtprojekts, kann allerdings für den Benutzer eine entscheidende Rolle bei der Such nach Items bieten.
Zudem versuchen wir die Inventarisierungs-App auch auf mobilen oder anderen Geräten zu unterstützen, dass Anwender nicht nur von ihrem Desktop auf die Website zugreifen können.
Das vorletzte Paket fügt wie sein Vorgänger neue Funktionen hinzu, allerdings auch die letzten wichtigen für den Anwender.
\begin{itemize}
\item
  Filterung und Sortierung (5\% des Gesamtprojekts)

  \begin{itemize}
  \item
    Nach Wunsch des Nutzers können Items nach bestimmten Eigenschaften gefiltert oder sortiert 			werden.
  \end{itemize}
  
\item
  Datenschutz und Anonymität (8\% des Gesamtprojekts)

  \begin{itemize}
  \item
    Der Datenschutz und die Anonymität umfassen die Bearbeitung von
    Profilen durch den Administrator, sowie die Verschlüsselung der
    jeweiligen Passwörter.
  \end{itemize}
  
\item
  Passwortwiederherstellung (5\% des Gesamtprojekts)

  \begin{itemize}
  \item
    Wenn das Passwort vergessen wird, ermöglicht diese Funktion die Wiederherstellung des 				Passworts, um wieder Zugang zu erlangen.
  \end{itemize}
  
\item
  Browsersupport (5\% des Gesamtprojekts)

  \begin{itemize}
  \item
    Der Browsersupport ermöglicht die Unterstützung von anderen Browsern, z.B. auf mobilen
    Geräten.
  \end{itemize}
  
\end{itemize}

\subsubsection{Abgabe Paket 4 (31\% des Gesamtprojekts)}
Abgabe Paket 4 beschreibt die Verbesserung und Optimierung der Inventarisierungs-App.
Im letzten Paket wird der Programmcode kommentiert damir er für Außenstehende leicht nachvollziehbar ist. Außerdem soll unser Projekt möglichst Fehlerunanfällig laufen.
Dies wird wie durch eine Validierung der Eingaben gewährleistet, dass bei einem  Technischen Versagen die Eingaben stets gesichert bleiben. 
Andere Sprachen werden ebenfalls hinzugefügt damit die Inventarisierungs-App von möglichst vielen Personen bedient werden kann.
Nach der Abgabe des letzten Paket stellt dies eine fertige Inventarisierungs-App bereit.
\begin{itemize}
\item
  Fehlerunanfälligkeit (8\% des Gesamtprojekts)

  \begin{itemize}
  \item
    Die Fehlerunanfälligkeit ermöglicht das Validieren von Eingaben. Zudem sorgt die Funktion 			dafür, dass bei technischem Versagen ein kleinst möglicher Schaden entsteht.
  \end{itemize}
  
\item
  Effiziente Implementierung (5\% des Gesamtprojekts)

  \begin{itemize}
  \item
    Die Funktion bietet die Möglichkeit, dass das Projekt mit geringer Laufzeit bei
    Prozessen arbeitet. Durch die effiziente Implementierung wird von der Website ein geringer 			Speicherbedarf genutzt. 
  \end{itemize}
  
\item
  Wartung (8\% des Gesamtprojekts)

  \begin{itemize}
  \item
    Der Programmcode soll gut dokumentiert und leicht strukturiert sein.
  \end{itemize}
  
\item
  Rollenverwaltung (5\% des Gesamtprojekts)

  \begin{itemize}
  \item
    Berechtigungen können hinzugefügt, bearbeitet oder entfernt werden,
    außerdem können die Administratoren die Verteilung von Anwenderrollen vornehmen.
  \end{itemize}
  
\item
  Mehrsprachigkeit (5\% des Gesamtprojekts)

  \begin{itemize}
  \item
    Es besteht die Option, andere Sprachen hinzuzufügen, um ein großes Publikum zu
    erreichen.
  \end{itemize}
  
\end{itemize}





\end{document}
