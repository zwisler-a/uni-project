\section{Projektplan}\label{projektplan}

\subsection{\emph{Inhaltsverzeichnis}}\label{inhaltsverzeichnis}

\subsubsection{Ziele}\label{ziele}

\begin{quote}
\begin{itemize}
\itemsep1pt\parskip0pt\parsep0pt
\item
  \emph{Muss/Kann Ziele sind aus den funktionalen und nicht funktionalen
  Anforderungen}
\item
  \emph{Insgesamt 18 Ziele}

  \begin{itemize}
  \itemsep1pt\parskip0pt\parsep0pt
  \item
    \emph{9 Muss Ziele 72\% des Gesamtprojekts}
  \item
    \emph{9 Kann Ziele 45\% des Gesamtprojekts}
  \end{itemize}
\end{itemize}
\end{quote}

\begin{itemize}
\item
  \textbf{Muss Ziele}
\item
  An- und Abmeldung
\item
  Datenzugriff
\item
  Dateneingabe
\item
  Typeneingabe
\item
  Administration
\item
  Datenschutz und Anonymität
\item
  Fehlerunanfälligkeit
\item
  Wartung
\item
  Benutzeroberfläche
\item
  \textbf{Kann Ziele}
\item
  Itemvorlage
\item
  Passwortwiederherstellung
\item
  Filterung und Sortierung
\item
  Verlinkung von Items
\item
  Rollenverwaltung
\item
  Browsersupport
\item
  Effizente Implementierung
\item
  Mehrsprachigkeit
\item
  Passwortänderung
\end{itemize}

\subsubsection{Arbeitspakete}\label{arbeitspakete}

\begin{itemize}
\item
  \textbf{Abgabe Paket 1}
\item
  An- und Abmeldung
\item
  Passwortänderung
\item
  Dateneingabe
\item
  Typeneingabe
\item
  Benutzeroberfläche
\item
  \textbf{Abgabe Paket 2}
\item
  Itemvorlage
\item
  Verlinkung von Items
\item
  Datenzugriff
\item
  Administration
\item
  \textbf{Abgabe Paket 3}
\item
  Filterung und Sortierung
\item
  Datenschutz und Anonymität
\item
  Passwortwiederherstellung
\item
  Browsersupport
\item
  \textbf{Abgabe Paket 4}
\item
  Fehlerunanfälligkeit
\item
  Effizente Implementierung
\item
  Wartung
\item
  Rollenverwaltung
\item
  Mehrsprachigkeit
\end{itemize}

\subsection{Beschreibung der
Arbeitspakete}\label{beschreibung-der-arbeitspakete}

\subsubsection{Abgabe Paket 1}\label{abgabe-paket-1}

\begin{quote}
\emph{37\% des Gesamtprojekts}
\end{quote}

\begin{itemize}
\itemsep1pt\parskip0pt\parsep0pt
\item
  An- und Abmeldung

  \begin{itemize}
  \itemsep1pt\parskip0pt\parsep0pt
  \item
    Diese Funktion bietet die Möglichkeit, sich über Nutzerdaten anzumelden, abzumelden
    oder sich zu registrieren, um neue Nutzerdaten zu erhalten.
    \textgreater{} \emph{8\% des Gesamtprojekts}
  \end{itemize}
\item
  Passwortänderung

  \begin{itemize}
  \itemsep1pt\parskip0pt\parsep0pt
  \item
    Diese Funktion ermöglicht das Passwort zu ändern.
    \textgreater{} \emph{5\% des Gesamtprojekts}
  \end{itemize}
\item
  Dateneingabe

  \begin{itemize}
  \itemsep1pt\parskip0pt\parsep0pt
  \item
    Diese Funktion ermöglicht das Entfernen, Bearbeiten, Hinzufügen und Anhängen neuer
    Items. \textgreater{} \emph{8\% des Gesamtprojekts}
  \end{itemize}
\item
  Typeneingabe

  \begin{itemize}
  \itemsep1pt\parskip0pt\parsep0pt
  \item
    Wie bei der Dateneingabe umfasst die Typeneingabe das Hinzufügen,
    Bearbeiten und Entfernen der jeweiligen Itemtypen. \textgreater{}
    \emph{8\% des Gesamtprojekts}
  \end{itemize}
\item
  Benutzeroberfläche

  \begin{itemize}
  \itemsep1pt\parskip0pt\parsep0pt
  \item
    Die Funktion ermöglicht ein übersichtliches Userinterface zu erstellen. \textgreater{}
    \emph{8\% des Gesamtprojekts}
  \end{itemize}
\end{itemize}

\subsubsection{Abgabe Paket 2}\label{abgabe-paket-2}

\begin{quote}
\emph{26\% des Gesamtprojekts}
\end{quote}

\begin{itemize}
\itemsep1pt\parskip0pt\parsep0pt
\item
  Itemvorlage

  \begin{itemize}
  \itemsep1pt\parskip0pt\parsep0pt
  \item
    Bei häufiger Verwendung von Items werden diese mithilfe dieser Funktion bereits hinterlegt
    und vorgeschlagen. \textgreater{} \emph{5\% des
    Gesamtprojekts}
  \end{itemize}
\item
  Verlinkung von Items

  \begin{itemize}
  \itemsep1pt\parskip0pt\parsep0pt
  \item
    Die Items werden untereinander verlinkt. \textgreater{} \emph{5\%
    des Gesamtprojekts}
  \end{itemize}
\item
  Datenzugriff

  \begin{itemize}
  \itemsep1pt\parskip0pt\parsep0pt
  \item
    Die Items werden in einer Tabelle erfasst und können mit dieser Funktion ebenfalls
    gesucht werden. \textgreater{} \emph{8\% des
    Gesamtprojekts}
  \end{itemize}
\item
  Administration

  \begin{itemize}
  \itemsep1pt\parskip0pt\parsep0pt
  \item
    Die Administration umfasst das Hinzufügen von Firmen oder
    Unternehmen, sowie die Typenverwaltung als auch das Hinzufügen von
    globalen Pflichtfeldern. \textgreater{} \emph{8\% des
    Gesamtprojekts}
  \end{itemize}
\end{itemize}

\subsubsection{Abgabe Paket 3}\label{abgabe-paket-3}

\begin{quote}
\emph{23\% des Gesamtprojekts}
\end{quote}

\begin{itemize}
\itemsep1pt\parskip0pt\parsep0pt
\item
  Filterung und Sortierung

  \begin{itemize}
  \itemsep1pt\parskip0pt\parsep0pt
  \item
    Nach Wunsch des Nutzers können Items nach bestimmten Eigenschaften gefiltert oder sortiert werden.
     \textgreater{} \emph{5\% des
    Gesamtprojekts}
  \end{itemize}
\item
  Datenschutz und Anonymität

  \begin{itemize}
  \itemsep1pt\parskip0pt\parsep0pt
  \item
    Der Datenschutz und die Anonymität umfassen die Bearbeitung von
    Profilen durch den Administrator, sowie die Verschlüsselung der
    jeweiligen Passwörter. \textgreater{} \emph{8\% des Gesamtprojekts}
  \end{itemize}
\item
  Passwortwiederherstellung

  \begin{itemize}
  \itemsep1pt\parskip0pt\parsep0pt
  \item
    Wenn das Passwort vergessen wird, ermöglicht diese Funktion die Wiederherstellung des Passworts, um wieder Zugang zu erlangen.
    \textgreater{} \emph{5\% des Gesamtprojekts}
  \end{itemize}
\item
  Browsersupport

  \begin{itemize}
  \itemsep1pt\parskip0pt\parsep0pt
  \item
    Der Browsersupport ermöglicht die Unterstützung von anderen Browsern, z.B. auf mobilen
    Geräten. \textgreater{} \emph{5\% des Gesamtprojekts}
  \end{itemize}
\end{itemize}

\subsubsection{Abgabe Paket 4}\label{abgabe-paket-4}

\begin{quote}
\emph{31\% des Gesamtprojekts}
\end{quote}

\begin{itemize}
\itemsep1pt\parskip0pt\parsep0pt
\item
  Fehlerunanfälligkeit

  \begin{itemize}
  \itemsep1pt\parskip0pt\parsep0pt
  \item
    Die Fehlerunanfälligkeit ermöglicht das Validieren von Eingaben. Zudem sorgt die Funktion dafür, dass bei technischem Versagen ein
    kleinst möglicher Schaden entsteht. \textgreater{} \emph{8\%
    des Gesamtprojekts}
  \end{itemize}
\item
  Effiziente Implementierung

  \begin{itemize}
  \itemsep1pt\parskip0pt\parsep0pt
  \item
    Die Funktion bietet die Möglichkeit, dass das Projekt mit geringer Laufzeit bei
    Prozessen arbeitet. Durch die effiziente Implementierung wird von der Website ein geringer Speicherbedarf genutzt. 
    \textgreater{} \emph{5\% des Gesamtprojekts}
  \end{itemize}
\item
  Wartung

  \begin{itemize}
  \itemsep1pt\parskip0pt\parsep0pt
  \item
    Der Programmcode soll gut dokumentiert und leicht strukturiert sein.
    \textgreater{} \emph{8\% des Gesamtprojekts}
  \end{itemize}
\item
  Rollenverwaltung

  \begin{itemize}
  \itemsep1pt\parskip0pt\parsep0pt
  \item
    Berechtigungen können hinzugefügt, bearbeitet oder entfernt werden,
    außerdem können die Administratoren die Verteilung von Anwenderrollen vornehmen. \textgreater{} \emph{5\%
    des Gesamtprojekts}
  \end{itemize}
\item
  Mehrsprachigkeit

  \begin{itemize}
  \itemsep1pt\parskip0pt\parsep0pt
  \item
    Es besteht die Option, andere Sprachen hinzuzufügen, um ein großes Publikum zu
    erreichen. \textgreater{} \emph{5\% des Gesamtprojekts}
  \end{itemize}
\end{itemize}
