\documentclass[11pt,a4paper]{report}

\usepackage[utf8]{inputenc}
\usepackage{titling}
\usepackage[german]{babel}
\usepackage[T1]{fontenc}
\usepackage{amsmath}
\usepackage{amsfonts}
\usepackage{amssymb}
\usepackage[left=3cm,right=2cm,top=2.5cm,bottom=2cm]{geometry}
\usepackage{graphicx}
\usepackage{fancyhdr}
\usepackage{color}
\usepackage[
colorlinks=true,
urlcolor=blue,
linkcolor=black
]{hyperref}
\pagestyle{fancy}

\lhead{Sina Opitz}
\chead{ak18b}
\rhead{13.01.2019}

\lfoot{}
\cfoot{\thepage}
\rfoot{}

\renewcommand{\headrulewidth}{0.4pt}
\renewcommand{\footrulewidth}{0.4pt}

\begin{document}
	\begin{titlepage}
		
		\pretitle{
			\vskip -3em
			\begin{figure}[h]
				\begin{center}
					\includegraphics[scale=0.55]{Logo.png}
				\end{center}
			\end{figure}
			\begin{center}
				\vskip -2em
				\large{Wintersemester 2018/19\\Softwaretechnikpraktikum} \vskip 9em
				\rule{5in}{0.4pt}\par \vskip 0.5em
			}
			\posttitle{\par\rule{5in}{0.4pt} \vskip 4em
				\Large Gruppe: ak18b \vskip 1.5em
				\normalsize Betreuer: Benjamin Lucas Friedland, Michael Fritz\vskip 1em
				\normalsize Gruppenmitglieder: Alexander Zwisler, Leon Kamuf, Leon Rudolph, Maurice Eisenblätter, Maximilian Gläfcke, Robin Seidel, Sina Opitz, Steve Woywod
		\end{center}}
		
		\title{\textbf{\Huge App zur Inventarisierung von Unternehmenswerten}\vskip 0.5em \huge Entwurfsbeschreibung}
		\date{}
		\maketitle
	\end{titlepage}
	\setcounter{secnumdepth}{4}
	\setcounter{tocdepth}{4}
	\tableofcontents
	\thispagestyle{empty}
	\newpage
	\setcounter{page}{1}
	\renewcommand\thesection{\arabic{section}}
	
	\section{Allgemeine Informationen}
	\subsection{Einführung}
Herzlich Willkommen!
\\~\\
Dieses Handbuch hilft Ihnen die Inventarisierungsapp NAME optimal zu nutzen.\\
Wir wünschen Ihnen viel Freude bei der Nutzung von NAME!\\
Am besten läuft die Webapp auf Google Chrome.
\\~\\
Änderungen vorbehalten\\
Version 1/11.03.19
	\subsection{Ihre Vorteile}
	\begin{itemize}
		\item intuitive Benutzeroberfläche
		\item für jedes Unternehmen geeignet
		\item vollständige digitale Inventarisierung
		\item übersichtliche Darstellung aller Gegenstände
		\item Erstellung von visuellen Übersichten
		\item Erstellen von Benutzerrollen zur Sicherung aller Gegenstände
		\item Gruppieren der Gegenstände nach Typen
	\end{itemize}
	\section{Erste Schritte}
	\textbf{Die App läuft vollständig im Browser. Es ist also keine Installation auf Ihrem Rechner nötig!}\\
	Empfohlener Browser: Google Chrome
	
	\begin{itemize}
		\item Rufen Sie im Browser \href{http://UNSEREADRESSE.de } auf
		\item Loggen Sie sich mit Ihrem Benutzernamen und Passwort an
		\item 
	\end{itemize}
	
	\section{Verwenden der Inventarisierungsapp}
	BILD DER STARTSEITE
	\subsection{Eine Gegenstandsgruppe hinzufügen}
	
	\subsection{Einen Gegenstand zu einer Gegenstandsgruppe hinzufügen}
	
	\section{weitere Funktionen für den Admin}
	\subsection{eine Benutzergruppe hinzufügen}
	
	\subsection{Rechte einer Benutzergruppe setzen}

	\subsection{Benutzer einer Benutzergruppe hinzufügen}
	
	\section{Erstellen von Übersichten}
	\subsection{Tabelle}
	
	\subsection{Balkendiagramm}
	
	\subsubsection{Tortendiagramm}}

	\section{Fehlerbehebung}
	LISTE ALLER ERRORS NÖTIG
	
	
	
	
	
	
	
	
\end{document}