\documentclass[12pt,a4paper]{report}
\usepackage[utf8]{inputenc}

\begin{document}
	\section*{Demo}
	Link auf Präsentation\\
	\subsection*{Itemtypen}
		\begin{itemize}
			\item Erstellen und Feldtypen erklären:\\
			Text: normaler Text, selbsterklärend (Bsp: Name)\\
			Nummer: normale Nummer (Integer), selbsterklärend (Bsp: Raumnummer)\\
			Wahr/Falsch: Wahr/Falsch-Feld, selbsterklärend (Bsp: defekt ja/nein)\\
			Verlinkung: Verlinkt Feldtyp mit einem anderen vorhandenen Feldtypen (Bsp: Laptop-Verlinkung auf bestimmten Raum)\\
			Farbe: RGB-Farbbereich, manuell wählbar von \texttt{\#000000} und \texttt{\#FFFFFF} (Bsp: schwarzer Laptop)\\
			Datei: Upload-Feld (Bsp: Rechnung zum Laptop hochladen)\\
			Datum: Datum kann aus Kalender ausgewählt werden (Bsp: Datum, wie lange der Laptop bereits im Raum ist)
			\item Benötigt (Markierung Feld ist Pflichtfeld)
			\item Einzigartig (Nur ein Objekt darf diesen Wert annehmen)
			\item zu jedem Feldtyp ein Feld erstellen:\\
			ein Typ mit Benötigt (Bsp: Name)\\
			ein Typ mit Einzigartig (Bsp: Nummer)\\
			\item auch kurz zeigen, die Feldtypen bearbeitet werden können
		\end{itemize}
	
	\subsection*{Item}
		\begin{itemize}
			\item Items zum Feldtypen erstellen
			\item auch Fehlermeldungen zeigen (Bsp: wenn zwei Laptops mit derselben Nummer erstellt wurden)
			\item Aufzeigen globales Feld
			\item Sortierfunktion zeigen
			\item Filterfunktion zeigen
			\item Suchfeld demonstrieren 
			\item auch kurz zeigen, wie Items geändert werden können
		\end{itemize}
	
	\subsection*{Firmen}
		\begin{itemize}
			\item neue Firma erstellen
			\item dort sollen auch Benutzer und Rollen erstellt werden
			\item Firmenauswahl oben rechts zeigen
		\end{itemize}

	\subsection*{Rollen}
		\begin{itemize}
			\item Rolle erstellen
			\item vorher Firma oben rechts auswählen
			\item Berechtigungen einstellen 
		\end{itemize}
	
	\subsection*{Benutzer}
		\begin{itemize}
			\item Benutzer anlegen
			\item Benutzer eine Rolle zuteilen
			\item Passwort zurücksetzen/ändern zeigen
			\item sich als neuer Benutzer einloggen und zeigen, dass dieser wirklich nur begrenzt Rechte durch Rollenzuweisung hat
			\item Daten bearbeiten (Stift unten links)
		\end{itemize}
	\clearpage
	\section*{Technische Umsetzung UI}
		UI mit Angular geschriebene Aufteilung in Module:
		\begin{itemize}
			\item Types (besitzt Store-Modul):\\
			Beinhaltet UI für Typenliste, Typendetails und globale Felder
			\item Items (besitzt Store-Modul):\\
			Beinhaltet UI für Itemliste, Itemdetails, Spaltenauswahl
			\item Roles (besitzt Store-Modul):\\
			Beinhaltet UI für Rollenliste, Details, Anlegen
			\item Company (besitzt Store-Modul):\\
			Beinhaltet UI für Unternehmen anlegen, Details
			\item User (besitzt Store-Modul):\\
			Beinhaltet UI für Benutzer anlegen, Details
			\item Permission:\\
			Das Permisson-Modul beinhaltet Helfer rund um die UI, damit Benutzer wirklich nur das sieht, was er darf\\
			\item Shell:\\
			Beinhaltet UI für die Navigation (links), Login-Fenster und Authorisierung
			\item Shared:\\
			Beinhaltet Inhalte, die von mehreren Modulen genutzt werden:\\
			Confirm Dialog: PopUp-Warnung bei DB-schädlicher Aktion (Bsp: löschen)\\
			Type Selector: Auswahl-Interface für ein Typ und Feld\\
			Default Page: Core-UI, auf der alles aufbaut (inkl. Elemente, die jede Seite besitzt, wie die Suchleiste)
			Store Factory: Alex, was heißt 'gleich' bei dir? xD
	\end{itemize}
\clearpage
	\section*{Datenhandling in der UI}
		\subsection*{Erklärung Store}
		Jede kontext-getrennte Ressource (Items/Types/Company/Roles/User) hat einen eigenen sog. Store.\\
		Dieser übernimmt die Backend Kommunikation für das Laden, Editieren und Löschen der Daten sowie das Anzeigen eines Fehlers bei fehlerhaften backend Kommunikation. \\
		(In einem Store gespeicherte Entitäten brauchen mindestens eine ID)\\~\\
		
		\subsection*{unsere Stores}
		Ein Store bietet dem "Benutzer" 6 Methoden:
		\texttt{
		\begin{itemize}
			\item load()
			\item byId(id)
			\item create(enitity)
			\item update(enitity)
			\item delete(id)
			\item store
		\end{itemize}}
	Enitäten in einem Store werden so lange nicht neu geladen bis entweder ein forced reload stattfindet (wechseln der Firma) oder die TTL (time to live) abläuft.\\
	Stores funktionieren über rxjs, was bedeutet ein Benutzer des Stores "abboniert" (subscribed) sich auf die gewünschten Daten und bekommt somit Änderungen im Store mit.\\
	Stores werden über das StoreFactory Modul erzeugt. Dies übernimmt das Einstellen von Standart Konfigurationen und das bereitstellen von anderen Services an den Store (bsp. TranslateService für Übersetzungen)
	\clearpage
	\section*{Lessons learned}
	\begin{itemize}
		\item Einstieg in Angular ziemlich schwer
		\item TypeScript nicht so fehlertollerant wie JavaScript
		\item Besser recherchieren vor und während des Projekts
		\item Reviewgespräche besser vorbereiten
		\item Zeit besser einteilen
		\item 
		\item 
		\item 
		\item 
		\item
	\end{itemize}
\end{document}