\documentclass[11pt,a4paper]{report}

\usepackage[utf8]{inputenc}
\usepackage{titling}
\usepackage[german]{babel}
\usepackage[T1]{fontenc}
\usepackage{amsmath}
\usepackage{amsfonts}
\usepackage{amssymb}
\usepackage[left=3cm,right=2cm,top=2.5cm,bottom=2cm]{geometry}
\usepackage{graphicx}
\usepackage{fancyhdr}
\pagestyle{fancy}

\lhead{Leon Kamuf}
\chead{ak18b}
\rhead{12.12.2018}

\lfoot{}
\cfoot{\thepage}
\rfoot{}

\renewcommand{\headrulewidth}{0.4pt}
\renewcommand{\footrulewidth}{0.4pt}

\begin{document}
\begin{titlepage}

\pretitle{%
\vskip -3em
\begin{figure}[h]
\begin{center}
\includegraphics[scale=0.55]{Logo.png}
\end{center}
\end{figure}
\begin{center}
\vskip -2em
\large{Wintersemester 2018/19\\Softwaretechnikpraktikum} \vskip 9em
\rule{5in}{0.4pt}\par \vskip 0.5em
}
\posttitle{\par\rule{5in}{0.4pt} \vskip 4em
\Large Gruppe: ak18b \vskip 1.5em
\normalsize Betreuer: Benjamin Lucas Friedland, Michael Fritz\vskip 1em
\normalsize Gruppenmitglieder: Alexander Zwisler, Leon Kamuf, Leon Rudolph, Maurice Eisenblätter, Maximilian Gläfcke, Robin Seidel, Sina Opitz, Steve Woywod
\end{center}}

\title{\textbf{\Huge App zur Inventarisierung von Unternehmenswerten}\vskip 0.5em \huge Lastenheft}
\date{}
\maketitle
\end{titlepage}
\setcounter{secnumdepth}{4}
\setcounter{tocdepth}{4}
\tableofcontents
\thispagestyle{empty}
\newpage
\setcounter{page}{1}
\renewcommand\thesection{\arabic{section}}

\section{Ausgangssituation}
In der Wirtschaft ist die Inventarisierung von Unternehmenswerten ein fester Bestandteil für einen organisierten Arbeitsablauf.
Ob man wissen möchte was ein Unternehmen für Reserven zum wirtschaften hat oder ob man unternehmensintern Arbeitsmaterialien sucht, dies alles kann man in der Inventarisierung finden.
Viele Unternehmen machen ihre Inventarisierung noch analog und nur wenige digital.
Die Arbeitswelt ist jedoch im Wandel und immer mehr Unternehmen wollen auf eine digitale Lösungen umsteigen, um eine zentrale Datenverwaltung zu haben, um Backups zu erstellen etc.
Es gibt bereits einige Apps um dies zu realisieren, allerdings sind diese meist auf spezielle Unternehmenstypen zugeschnitten.

\vskip 8em
\section{Zielsetzung und Produkteinsatz}
\subsection{Vision}
Unternehmen sollen mit Hilfe der entstehenden App unternehmensintern besser organisieren können.
Ziel des Projektes ist also eine leicht bedienbare Organisationshilfe.
Als Alternative zur digitalen Lösung gibt es noch die analoge Option, welche jedoch nicht so zentral anwendbar ist, wie die Digitale.
Die anderen Apps, welche auf dem Markt sind, sind meist nur auf einen Unternehmenstypen spezialisiert.
Die bestehenden Apps sind, aufgrund ihrer Spezialisierung auf ein Unternehmen, des Öfteren nicht besonders Wandlungsfähig und daher nicht zentral, für jedes Unternehmen, anwendbar.
Aus diesem Grund soll eine App konzipiert werden, zur Inventarisierung von Unternehmenswerten, welche hoch konfigurierbar ist.
\subsection{Zielsetzung}
Die zu entwickelnde Software soll eine organisatorische Erleichterung für Unternehmen sein, daher soll sie Leicht und intuitiv zu bedienen sein.
Die Listung der Items eines Unternehmens soll entsprechend anschaulich sein, vor Allem für die Beziehung von mehreren Unternehmen unter einer Dachgesellschaft.
Eine hohe Konfigurierbarkeit ist ebenfalls anzustreben, damit alle Eventualitäten abgedeckt werden können.
Es sollen Nutzerrollen festgelegt werden, damit nicht jeder Nutzer alles machen kann.
Des weiteren sollen Items Dokumente angehangen werden können.
\subsection{Produkteinsatz}
Das Produkt soll in einem Bürounternehmen eingesetzt werden.
Jede Person aus dem Unternehmen soll die App nutzen können, weshalb die Bedienbarkeit und Nutzerrollen besonders wichtig sind.
Die App dient als zentrales Organisationstool, um schnell einen Überblick über die Objekte im Unternehmen zu erhalten.
\newpage

\section{Funktionale Anforderungen}

\subsection{Muss-Ziele}
\subsubsection{An- und Abmeldung}

\textbf{/LF0010/} Login
\par
\begingroup
\leftskip=1cm
\noindent Der Anwender der App soll sich mit seinem Usernamen, sowie mit seinem Passwort, anmelden können.\\
\par
\endgroup

\leftskip=-1.5em
\textbf{/LF0020/} Logout
\par
\begingroup
\leftskip=1cm
\noindent Der Anwender soll sich jederzeit ausloggen können.\\
\par
\endgroup

\textbf{/LF0030/} Registrieren neuer Anwender
\par
\begingroup
\leftskip=1cm
\noindent Es soll die Möglichkeit bestehen neue Nutzer der App hinzuzufügen und ihnen ihre Daten zur Anmeldung zu übermitteln.
Dabei muss darauf geachtet werden dass das Passwort mindestens 8 Zeichen lang sein soll, des Weiteren soll es mindestens einen Klein- und einen Großbuchstaben enthalten, sowie mindestens ein Sonderzeichen und eine Zahl.\\
\par
\endgroup

\textbf{/LF0040/} Wiederherstellung des Passworts
\par
\begingroup
\leftskip=1cm
\noindent Für den Fall, dass ein Anwender sein Passwort vergisst, soll es die Option geben ein neues zu erhalten, welches ebenfalls die Passwortkriterien erfüllt.
(siehe /LF0030/)\\
\par
\endgroup

\leftskip=0em
\subsubsection{User Self Care}

\textbf{/LF0110/} E-Mail Adresse ändern
\par
\begingroup
\leftskip=1cm
\noindent Für den Anwender soll die Option bestehen, seine hinterlegte E-Mail Adresse jederzeit ändern zu können.\\
\par
\endgroup

\leftskip=-1.5em
\textbf{/LF0120/} Passwort ändern
\par
\begingroup
\leftskip=1cm
\noindent Der Anwender soll sein Passwort ändern können, solang er eingeloggt ist.
Das geänderte Passwort muss jedoch ebenfalls die Kriterien für ein Passwort erfüllen.
(siehe /LF0030/)\\
\par
\endgroup

\leftskip=0em
\subsubsection{Datenzugriff}
\textbf{/LF0210/} Anzeigen aller erfassten Items
\par
\begingroup
\leftskip=1cm
\noindent Alle erfassten Items sollen als Tabelle ausgegeben werden.
\begin{enumerate}
\leftskip=3em
\item[a)] Benutzerdefinierte Spalten
\item[] Die anzuzeigenden Spalten können von jedem Anwender individuell definiert werden.
\end{enumerate}
\par
\endgroup

\leftskip=-1.5em
\textbf{/LF0220/} Suche
\par
\begingroup
\leftskip=1cm
\noindent Es soll ein Suchfeld vorhanden sein, womit es einem Anwender gestattet ist gezielt nach Items, mit der vollständigen Bezeichnung, zu suchen.
\begin{enumerate}
\leftskip=3em
\item[a)] Typspezifische Suche
\item[] Es soll nach einem spezifischen Typ gesucht werden können, wie zum Beispiel nach dem Typ $"$Laptop$"$.
\item[b)]Feldspezifische Suche
\item[] Es soll die Suche nach bestimmten Werten in Feldern ermöglicht sein, zum Beispiel die Suche nach "Benutzern$"$ mit dem Wert $"$Max Mustermann$"$.
\end{enumerate}
\par
\endgroup

\textbf{/LF0230/} Sortieren von Items
\par
\begingroup
\leftskip=1cm
\noindent Die Sortierung von Items nach dem Nutzerwunsch soll möglich sein.\\
\par
\endgroup

\leftskip=0em
\subsubsection{Dateneingabe}
\textbf{/LF0310/} Neue Items hinzufügen
\par
\begingroup
\leftskip=1cm
\noindent Der Anwender soll neue Items hinzufügen können, mit den, passend zum Itemtyp, entsprechenden Eigenschaften.\\
\par
\endgroup

\leftskip=0em

\leftskip=-1.5em
\textbf{/LF0320/} Anhang von Dokumenten
\par
\begingroup
\leftskip=1cm
\noindent Es soll möglich sein einem Item Dateien (PDF, PNG etc.) versioniert anhängen zu können.\\
\par
\endgroup

\leftskip=-1.5em
\textbf{/LF0330/} Bearbeiten von Items
\par
\begingroup
\leftskip=1cm
\noindent Der Anwender soll die Möglichkeit haben bereits vorhandene Items überarbeiten zu können.\\
\par
\endgroup

\leftskip=-1.5em
\textbf{/LF0340/} Entfernen von Items
\par
\begingroup
\leftskip=1cm
\noindent Es soll für den Anwender möglich sein, Items restlos löschen zu können.\\
\par
\endgroup

\textbf{/LF0350/} Eingabehilfe für Itemfelder
\par
\begingroup
\leftskip=1cm
\noindent Beim Hinzufügen, beziehungsweise beim Bearbeiten, von Felder in einem Item, solllen bei bestimmten Feldertypen Eingabehilfen angeboten werden.
So soll, zum Beispiel bei der Eingabe eines Farbwertes, ein Color-Picker erscheinen oder bei der Eingabe eines Dokuments ein Fileupload Dialog.\\
\par
\endgroup

\textbf{/LF0360/} Verlinkung von Items
\par
\begingroup
\leftskip=1cm
\noindent Es soll möglich sein Items untereinander zu verlinken, zum Beispiel existiert ein Item $"$Location X$"$ , welches ein Feld von Item $"$Y$"$ ist.
Dies kann auf Orte, Personen etc. übertragen werden.\\
\par
\endgroup

\leftskip=0em
\subsubsection{Typeneingabe}

\textbf{/LF0410/} Hinzufügen neuer Itemtypen
\par
\begingroup
\leftskip=1cm
\noindent Es soll möglich sein neue Typen von Items hinzufügen zu können.\\
\par
\endgroup

\leftskip=-1.5em
\textbf{/LF0420/} Bearbeitung von Itemtypen
\par
\begingroup
\leftskip=1cm
\noindent Der Anwender soll Typen überarbeiten können, zu dem sollen auch alle Items, welche dem Typ zugehörig sind, aktualisiert werden.\\
\par
\endgroup

\leftskip=-1.5em
\textbf{/LF0430/} Pflichfelder von Itemtypen
\par
\begingroup
\leftskip=1cm
\noindent Itemtypen sollen, falls angebracht, Pflichtfelder zugeschrieben bekommen, welche bei jedem neuem Item des Typs vorhanden sein müssen.\\
\par
\endgroup

\leftskip=-1.5em
\textbf{/LF0440/} Löschen von Itemtypen
\par
\begingroup
\leftskip=1cm
\noindent Das Löschen von bestimmten Itemtypen soll implementiert sein.\\
\par
\endgroup

\newpage
\leftskip=0em
\subsubsection{Administration}

\textbf{/LF0510/} Rollenverwaltung
\par
\begingroup
\leftskip=1cm
\noindent Es soll, vorhandenen Anwenderrollen, Berechtigungen hinzugefügt, bzw. entfernt werden können.
Als mögliche Anwenderrollen sind die folgenden Beispiele vorgesehen.\\
\begin{enumerate}
\leftskip=3em
\item[a)] \textbf{Admin:} Alle Berechtigungen, in Hinsicht auf hinzufügen, bearbeiten und löschen.
Außerdem Berechtigung zur Verwaltung von Dachgesellschaften.\\
\item[b)] \textbf{Unternehmensverwalter:} Genau wie Admin, hat diese Nutzerrolle alle Berechtigungen zum Hinzufügen.
Bearbeiten und Löschen, jedoch kann er, für den Fall der Dachgesellschaftsbeziehung, nur auf das ihm zugewiesene Unternehmen zugreifen.\\
\item[c)] \textbf{Nutzer:} Nur Berechtigung zum Bearbeiten von Items.\\
\end{enumerate}
\par
\endgroup

\leftskip=-1.5em
\textbf{/LF0520/} Löschen von Profilen
\par
\begingroup
\leftskip=1cm
\noindent Der Admin soll, genau wie bei der Bearbeitung von Profildaten, stets die Möglichkeit haben Profile zu löschen, um Personen, welche nicht mehr zum Unternehmen gehören, jederzeit entfernen zu können.\\
\par
\endgroup

\leftskip=-1.5em
\textbf{/LF0530/} Firmenverwaltung
\par
\begingroup
\leftskip=1cm
\noindent Es soll möglich sein neue Unternehmen/Firmen der App hinzuzufügen und zu bearbeiten, sowie mehrere Unternehmen/Firmen unter einer Dachgesellschaft zusammenzufügen.\\
\par
\endgroup

\leftskip=-1.5em
\textbf{/LF0540/} Typenverwaltung
\par
\begingroup
\leftskip=1cm
\noindent Es soll das Hinzufügen, Bearbeiten und Löschen von Typen möglich sein. (siehe Typeneingabe)\\
\par
\endgroup

\leftskip=-1.5em
\textbf{/LF0550/} Globale Pflichtfelder
\par
\begingroup
\leftskip=1cm
\noindent Es soll das Hinzufügen und Bearbeiten von Feldern ermöglicht sein, welche in jedem Item vorkommen.\\
\par
\endgroup

\leftskip=0em
\subsection{Kann-Ziele}

\textbf{/LF0610/} Mehrsprachigkeit
\par
\begingroup
\leftskip=1cm
\noindent Die Anwendung soll in mehreren Sprachen verfügbar sein und Übersetzungen sollen einfach hinzugefügt werden können.\\
\par
\endgroup

\leftskip=-1.5em
\textbf{/LF0620/} Itemtyp Vorlagen
\par
\begingroup
\leftskip=1cm
\noindent Typische Itemtypen, welche oft in Unternehmen vorhanden sind,
sollen als Vorlage bereits zur Verfügung stehen, um das Aufsetzen der Software zu vereinfachen.\\
\par
\endgroup

\leftskip=-1.5em
\textbf{/LF0630/} Statistiken für Itembestände
\par
\begingroup
\leftskip=1cm
\noindent Für den Anwender soll es möglich sein eine einfache Statistik aufzurufen, welche die Anzahl der Items, von bestimmten Itemtypen, widerspiegelt.\\
\par
\endgroup

\leftskip=-1.5em
\textbf{/LF0640/} Benutzerhandbuch
\par
\begingroup
\leftskip=1cm
\noindent In der Anwendung, soll es die Möglichkeit geben, ein Benutzerhandbuch zu öffnen, sollte es bei der Bedienung zu Fragen kommen.\\
\par
\endgroup

\leftskip=-1.5em
\textbf{/LF0650/} Chat
\par
\begingroup
\leftskip=1cm
\noindent Als zusätzliche Kommunikationsmöglichkeit, soll ein einfaches Chattool implementiert werden, damit kleine Nachrichten/Anmerkungen nicht als separate E-Mail verfasst werden müssen.\\
\par
\endgroup

\leftskip=-1.5em
\textbf{/LF0660/} Nutzerverwaltung
\par
\begingroup
\leftskip=1cm
\noindent Der Admin soll Berechtigung haben bestehende Nutzerrollen in ihren Berechtigungen anzupassen, sowie neue Nutzerrollen anzulegen und ihnen entsprechende Berechtigungen zuzuweisen.\\
\par
\endgroup

\vskip 8em
\leftskip=0em
\section{Nicht-funktionale Anforderungen}
\subsection{Muss-Ziele}
\subsubsection{Datenschutz und Anonymität}

\textbf{/LN0010/} Sicherheit der Anmeldedaten
\par
\begingroup
\leftskip=1cm
\noindent Um die Sicherheit der Anmeldedaten zu gewährleisten, sollen diese verschlüsselt gespeichert werden.
Die Verschlüsselung erfolgt über hashen.\\
\par
\endgroup

\leftskip=0em
\subsubsection{Fehlerunanfälligkeit}

\textbf{/LN0110/} Validieren von Benutzereingaben
\par
\begingroup
\leftskip=1cm
\noindent Sämtliche Benutzereingaben müssen validiert werden, sollte die Validierung fehlschlagen, darf die Eingabe nicht übernommen werden.\\
\par
\endgroup

\leftskip=-1.5em
\textbf{/LN0120/} Datensicherheit
\par
\begingroup
\leftskip=1cm
\noindent Sollte es zu internen Fehlern kommen, ist der Schaden (Datenverlust etc.) so gering wie möglich zu halten.
Ein Absturz der App sollte unter allen Umständen vermieden werden.\\
\par
\endgroup

\leftskip=0em
\subsubsection{Benutzeroberfläche}

\textbf{/LN0310/} Intuitive grafische Interaktion
\par
\begingroup
\leftskip=1cm
\noindent Das UI (User-Interface) soll so einfach wie möglich gehalten werden, damit es intuitiv von dem Benutzer verwendet werden kann.\\
\par
\endgroup

\leftskip=0em
\subsection{Kann-Ziele}
\textbf{/LN0410/} Browsersupport
\par
\begingroup
\leftskip=1cm
\noindent Das Programm soll von so vielen Browsern wie möglich unterstützt werden.\\
\par
\endgroup

\newpage
\leftskip=0em
\section{Qualitätsmatrix nach ISO 25010}
	\begin{center}
		\begin{tabular}{|l|c|c|c|c|}
		\hline
		& hoch & mittel & niedrig& nicht anwendbar\\
		\hline
		Funktionalität  &       X      &              & 		&\\
		Zuverlässigkeit	&              &      X       & 		&\\
		Effizienz 		&              &              & 	X	&\\
		Sicherheit  	&       X      &              & 		&\\
		Kompatibilität  &              &              & 	X	&\\
		Benutzbarkeit  	&       X      &              & 		&\\
		Wartbarkeit  	&              &      X       & 		&\\
		Portierbarkeit  &              &      X       & 		&\\
		\hline
		\end{tabular}
 	\end{center}
\vskip 1.5em
Besonders hohe Priorität hat bei uns die Sicherheit.
Unternehmenswerte und Dokumente sollen ausschließlich von autorisierten Personen gesehen und bearbeitet werden können, da es sich hierbei um sehr sensible Daten handeln kann. Ebenfalls wichtig ist die Benutzbarkeit, da jeder Mitarbeiter die App nutzen soll und deshalb Schulungen für die Benutzung nicht notwendig sein sollen.
Effizienz hingegen wird von uns als eher unwichtig eingestuft, da die Mengen an Daten überschaubar bleiben und keine rechenintensiven Berechnungen Teil der App sind.
Die Kompatibilität haben wir ebenfalls niedrig eingestuft, da wir vorerst nur den Chrome-Browser unterstützen möchten.

\vskip 8em
\section{Lieferumfang und Abnahmekriterien}

\subsection{Lieferumfang}
Geliefert werden soll eine Webanwendung zur Verwaltung unternehmensspezifischer Werte, welche wesentliche Funktionalitäten bereitstellt.
Sie soll speziell für Bürounternehmen konzipiert sein.
Es soll die Benutzeroberfläche, sowie die Nutzerverwaltung, bereits implementiert sein.
Das Back-End wird durch $"$MariaDB$"$, sowie $"$SQL$"$, realisiert, um alle nötigen Informationen speichern, verwalten und synchronisieren zu können.
Des Weiteren wird eine vollständige Dokumentation zur Verfügung gestellt.

\subsection{Abnahmekriterien}
Als Hauptabnahmekriterium wird ein vollständig funktionstüchtiger Prototyp des Inventarisierungstools festgelegt.
Es sollen alle Muss-Ziele und so viele Kann-Ziele wie möglich fertiggestellt sein.
Der Lieferumfang soll eingehalten werden.
Neben der vollständigen Dokumentation das Projekts, sollte darauf geachtet werden, dass die Qualitätsmatrix eingehalten wird.

\vskip 8em
\section{Vorprojekt}
Im Rahmen des Vorprojekts soll das wesentliche Grundgerüst, bestehend aus Datenbank und Server- und Client-Grundkonzept, erstellt werden.
Zu diesem Zeitpunkt sollte das Hinzufügen und Löschen von Items, sowie deren Darstellung, möglich sein.
Die Darstellung der Items sollte am besten in Form einer Liste vorgenommen werden.
Es muss ein Grundkonzept für diese Items bereits konzipiert und implementiert sein. Dachgesellschaften, sowie die Darstellung und Modifizierung der Items von den dazugehörigen Unternehmen, sollen ebenfalls fertiggestellt sein.
Die Rechtevergabe an einzelne Personengruppen, damit Diese nicht alles einsehen können, sondern nur das was vom Admin für sie freigegeben ist, soll auch funktionstüchtig sein.

\vskip 8em
\section{Glossar}
\textbf{Inventar}
\par
\begingroup
\leftskip=1cm
\noindent ist die Gesamtheit der zu einem Betrieb/Unternehmen gehörenden
Einrichtungsgegenstände und Vermögenswerten.
Zudem muss das Inventar jährlich inventarisiert werden.\\
\par
\endgroup

\leftskip=-1.5em
\textbf{Inventarisierung}
\par
\begingroup
\leftskip=1cm
\noindent ist die Bestandsaufnahme von unternehmensspezifischen Objekten
in Hinsicht auf spezielle Merkmale selbiger.
Sie bildet die Informationsgrundlage für
Kosten- und Leistungsrechnungen sowie zur Ermittlung von einem potentiellen Reinvestitionsbedarf.\\
\par
\endgroup

\textbf{Item}
\par
\begingroup
\leftskip=1cm
\noindent ist die Bezeichnung der konkreten, unternehmensspezifischen Objekten, welche zu einem Unternehmen gehören.\\
\par
\endgroup

\textbf{Feld}
\par
\begingroup
\leftskip=1cm
\noindent ist der Begriff für die Merkmale eines Items und beinhaltet Metainformationen,
wie zum Beispiel ob es ein Pflichfeld ist und von welchem Datentyp es ist.\\
\par
\endgroup

\textbf{Itemtyp}
\par
\begingroup
\leftskip=1cm
\noindent ist das Grundgerüst, welches ein Item beschreibt.
Der Itemtyp beinhaltet alle Felder, welche ein Item haben kann.\\
\par
\endgroup

\textbf{Dachgesellschaft}
\par
\begingroup
\leftskip=1cm
\noindent beschreibt ein Wirtschaftskonzept, bei dem die Dachgesellschaft (in Literatur auch Holdinggesellschaft) als Verwaltung der zugehörigen Unternehmen dient.\\
\par
\endgroup

\end{document}
