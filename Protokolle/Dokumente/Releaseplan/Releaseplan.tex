\documentclass[11pt,a4paper]{report} 
\usepackage[utf8]{inputenc}
\usepackage{titling}
\usepackage[german]{babel}
\usepackage[T1]{fontenc}
\usepackage{amsmath}
\usepackage{csquotes}
\usepackage{ulem}
\usepackage[dvipsnames]{xcolor}
\usepackage{amsfonts}
\usepackage{amssymb}
\usepackage[left=3cm,right=2cm,top=2.5cm,bottom=2cm]{geometry}
\usepackage{graphicx}
\usepackage{fancyhdr}
\usepackage{color}
\usepackage[
colorlinks=true,
urlcolor=blue,
linkcolor=black
]{hyperref}
\pagestyle{fancy}

\lhead{Leon Kamuf}
\chead{ak18b}
\rhead{28.01.2019}

\lfoot{}
\cfoot{\thepage}
\rfoot{}

\renewcommand{\headrulewidth}{0.4pt}
\renewcommand{\footrulewidth}{0.4pt}

\begin{document}
	\begin{titlepage}		
		\pretitle{
			\vskip -3em
			\begin{figure}[h]
				\begin{center}
				\includegraphics[scale=0.55]{Logo.png}
				\end{center}
			\end{figure}
			\begin{center}
				\vskip -2em
				\large{Wintersemester 2018/19\\Softwaretechnikpraktikum} \vskip 9em
				\rule{5in}{0.4pt}\par \vskip 0.5em
			}
			\posttitle{\par\rule{5in}{0.4pt} \vskip 4em
				\Large Gruppe: ak18b \vskip 1.5em
				\normalsize Betreuer: Benjamin Lucas Friedland, Michael Fritz\vskip 1em
				\normalsize Gruppenmitglieder: Alexander Zwisler, Leon Kamuf, Leon Rudolph, Maurice Eisenblätter, Maximilian Gläfcke, Robin Seidel, Sina Opitz, Steve Woywod
		\end{center}}
		
		\title{\textbf{\Huge App zur Inventarisierung von Unternehmenswerten}\vskip 0.5em \huge Releaseplan}
		\date{}
		\maketitle
	\end{titlepage}
	\setcounter{secnumdepth}{4}
	\setcounter{tocdepth}{4}
	\tableofcontents
	\thispagestyle{empty}
	\newpage
	\setcounter{page}{1}
	\renewcommand\thesection{\arabic{section}}


\section{Arbeitspaket 1}
Release: 14.01.2019\\\\
Das Vorprojekt bildet das wesentliche Grundgerüst der Anwendung. Es basiert auf dem Server- Client-Grundkonzept.\\
Es beinhaltet eine MariaDb Datenbank, eine REST-Schnittstelle sowie ein Frontend.\\
Weiterhin wird für das Frontend eine App-Shell entwickelt, welche das generelle Layout widerspiegelt. Eine MariaDb mit REST Api wird auf dem Praktikumsserver eingerichtet.
Das Vorprojekt soll zur Einarbeitung in die verschiedenen Technologien dienen. 

\subsection{Umsetzung}
Im Vorprojekt werden alle essentiellen Funktionalitäten umgesetzt:
\begin{itemize}
\item Login /LF0010/
\item Logout /LF0020/
\item Hinzufügen neuer Items /LF0310/
\item Bearbeiten von Items /LF0330/
\item Entfernen von Items /LF0340/
\item Eingabehilfe für Itemfelder /LF0350/
\item Hinzufügen neuer Itemtypen /LF0410/
\item Pflichtfelder von Itemtypen /LF0430/
\item Löschen von Itemtypen /LF0440/
\item Mehrsprachigkeit /LF0610/
\item Sicherheit der Anmeldedaten /LN0010/
\item Validieren von Benutzereingaben /LN0110/
\item Datensicherheit /LN0120/
\item Intuitive grafische Interaktion /LN0310/
\item Browsersupport /LN0410/
\end{itemize}

\newpage

\section{Arbeitspaket 2}
Release: 28.01.2019
\subsection{Umsetzung}
Im zweiten Arbeitspaket ist das Hauptthema die Administration von Unternehmen.
\begin{itemize}
\item \textbf{Wiederherstellung des Passworts /LF0040/}
\item \textbf{E-Mail Adresse ändern /LF0110/}
\item \textbf{Passwort ändern /LF0120/}
\item Anzeigen aller erfassten Items /LF0210/
\item Suche /LF0220/
\item Sortieren von Items /LF0230/
\item \sout{Verlinkung von Items /LF0360/}
\item Bearbeitung von Itemtypen /LF0420/
\item \sout{Rollenverwaltung /LF0510/}
\item Löschen von Profilen /LF0520/
\item Firmenverwaltung /LF0530/
\item Typenverwaltung /LF0540/
\item \sout{Globale Pflichtfelder /LF0550/}
\end{itemize}

\subsection{Status}
Implementiert:
\begin{itemize}
\item Anzeigen aller erfassten Items /LF0210/
\item Suche /LF0220/
\item Sortieren von Items /LF0230/
\item Bearbeitung von Itemtypen /LF0420/
\item Löschen von Profilen /LF0520/
\item Firmenverwaltung /LF0530/
\item Typenverwaltung /LF0540/\\
\end{itemize}
Aus \enquote{\hskip0.1em Arbeitspaket 3\hskip0.1em} übernommen und implementiert:
\begin{itemize}
\item Wiederherstellung des Passworts /LF0040/
\item E-Mail Adresse ändern /LF0110/
\item Passwort ändern /LF0120/\\
\end{itemize}

\newpage

In \enquote{\hskip0.1em Arbeitspaket 3\hskip0.1em} verschoben:
\begin{itemize}
\item Verlinkung von Items /LF0360/
\item Rollenverwaltung /LF0510/
\item Globale Pflichtfelder /LF0550/
\end{itemize}



\section{Arbeitspaket 3}
Release: 04.02.2019
\subsection{Umsetzung}
Im dritten Arbeitspaket wird sich hauptsächlich mit der Dokumentenverwaltung beschäftigt.

\begin{itemize}
\item Registrieren neuer Anwender /LF0030/
\item \sout{Wiederherstellung des Passworts /LF0040/}
\item \sout{E-Mail Adresse ändern /LF0110/}
\item \sout{Passwort ändern /LF0120/}
\item Anhang von Dokumenten /LF0320/
\item \textbf{Verlinkung von Items /LF0360/}
\item \textbf{Rollenverwaltung /LF0510/}
\item \textbf{Globale Pflichtfelder /LF0550/}
\end{itemize}

\subsection{Status}
Aus \enquote{\hskip0.1em Arbeitspaket 2\hskip0.1em} übernommen:
\begin{itemize}
\item Verlinkung von Items /LF0360/
\item Rollenverwaltung /LF0510/
\item Globale Pflichtfelder /LF0550/\\
\end{itemize}
In \enquote{\hskip0.1em Arbeitspaket 2\hskip0.1em} verschoben:
\begin{itemize}
\item Wiederherstellung des Passworts /LF0040/
\item E-Mail Adresse ändern /LF0110/
\item Passwort ändern /LF0120/
\end{itemize}

\newpage

\section{Arbeitspaket 4}
Release: 04.03.2019
\subsection{Umsetzung}
Im vorletzten Arbeitspaket werden eventuelle Restbestände der funktionalen Anforderungen nachgebessert oder fertiggestellt.
\begin{itemize}
\item Itemtyp Vorlagen /LF0620/
\item Nutzerverwaltung /LF0660/
\end{itemize}

\section{Arbeitspaket 5}
Release 11.03.2019
\subsection{Umsetzung}
Vor Abgabe des finalen Releases werden abschlie{\ss}ende Tests durchgeführt. Optional können Beta-User die Software testen um eventuelle Bugs und Ineffizienzen aufzuspüren.
\begin{itemize}
\item Statistiken für Itembestände /LF0630/
\item Benutzerhandbuch /LF0640/
\item Chat /LF0650/
\end{itemize}

\end{document}
