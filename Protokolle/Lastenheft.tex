\documentclass[11pt,a4paper]{report}

\usepackage[utf8]{inputenc}
\usepackage{titling}
\usepackage[german]{babel}
\usepackage[T1]{fontenc}
\usepackage{amsmath}
\usepackage{amsfonts}
\usepackage{amssymb}
\usepackage[left=3cm,right=2cm,top=2.5cm,bottom=2cm]{geometry}
\usepackage{graphicx}
\usepackage{fancyhdr}
\pagestyle{fancy}

\lhead{Leon Kamuf}
\chead{ak18b}
\rhead{22.11.2018}

\lfoot{}
\cfoot{\thepage}
\rfoot{}

\renewcommand{\headrulewidth}{0.4pt}
\renewcommand{\footrulewidth}{0.4pt}

\begin{document}
\begin{titlepage}

\pretitle{%
\vskip -3em
\begin{figure}[h]
\begin{center}
\includegraphics[scale=0.55]{Logo.png}
\end{center}
\end{figure}
\begin{center}
\vskip -2em
\large{Wintersemester 2018/19\\Softwaretechnikpraktikum} \vskip 9em
\rule{5in}{0.4pt}\par \vskip 0.5em
}
\posttitle{\par\rule{5in}{0.4pt} \vskip 4em
\Large Gruppe: ak18b \vskip 1.5em
\normalsize Betreuer: Benjamin Lucas Friedland, Michael Fritz\vskip 1em
\normalsize Gruppenmitglieder: Alexander Zwisler, Leon Kamuf, Leon Rudolph, Maurice Eisenblätter, Maximilian Gläfcke, Robin Seidel, Sina Opitz, Steve Woywod
\end{center}}

\title{\textbf{\Huge App zur Inventarisierung von Unternehmenswerten}\vskip 0.5em \huge Lastenheft}
\date{}
\maketitle
\end{titlepage}
\setcounter{secnumdepth}{4}
\setcounter{tocdepth}{4}
\tableofcontents
\thispagestyle{empty}
\newpage
\setcounter{page}{1}
\renewcommand\thesection{\arabic{section}}

\section{Ausgangssituation}
In der Wirtschaft ist die Inventarisierung von Unternehmenswerten ein fester Bestandteil für einen organisierten Arbeitsablauf.
Ob man wissen möchte was ein Unternehmen für Reserven zum wirtschaften hat oder ob man unternehmensintern Arbeitsmaterialien sucht, dies alles kann man in der Inventarisierung finden.
Viele Unternehmen machen ihre Inventarisierung noch analog und nur wenige digital.
Die Arbeitswelt ist jedoch im Wandel und immer mehr Unternehmen wollen auf eine digitale Lösungen umsteigen.
Es gibt bereits einige Apps um dies zu realisieren, allerdings sind diese meist auf spezielle Unternehmenstypen zugeschnitten.

\vskip 10em
\section{Zielsetzung und Produkteinsatz}
\subsection{Vision}
Unternehmen sollen mit Hilfe der entstehenden App unternehmensintern besser organisieren können.
Ziel des Projektes ist also eine leicht bedienbare Organisationshilfe.
Wie bereits erwähnt gibt es noch die analoge Option, welche jedoch nicht so zentral anwendbar ist, wie eine digitale Lösung.
Die anderen Apps, welche auf dem Markt sind, sind meist nur auf einen Unternehmenstypen spezialisiert.
Aus diesem Grund soll eine App konzipiert werden, zur Inventarisierung von Unternehmenswerten, welche die Inventarisierung nicht als organisatorisch Last erscheinen lässt, sondern, durch ihre leichte Bedienbarkeit, den Arbeitsalltag erleichtert.
\subsection{Zielsetzung}
Die zu entwickelnde Software soll eine organisatorische Erleichterung für Unternehmen werden.
Um dieses Ziel zu erreichen wird eine leichte, intuitive Bedienbarkeit angestrebt, um zusätzliche Schulungen für die Software zu vermeiden.
Die entsprechende Listung und Verwaltung der Software, für ein Unternehmen, sollte entsprechend anschaulich und einfach sein.
Unternehmen, welche sich unter einer Dachgesellschaft zusammenbringen, sind häufig anzutreffen in der freien Wirtschaft, weshalb die Software eine Möglichkeit zur Organisation dieser Beziehung haben sollte.
Selbstverständlich darf nicht jede Person alles machen, deshalb werden mehrere Rollen eingeteilt, welche verschiedene Berechtigungen zur Bearbeitung, beziehungsweise zur Einsicht in diverse Dokumente haben.
Den gelisteten Items können Dokumente angehangen werden.
\subsection{Produkteinsatz}
Zielgruppe der Software sind Unternehmen, welche Inventarisierungsbedarf haben.
Es sollen Arbeitgeber angesprochen werden, welche die Software an ihre Arbeitnehmer weiterleiten.
Die Arbeitnehmer sollen dann die Benutzung von Arbeitsmaterialien dokumentieren, damit die Materialien schnell gefunden werden können, sollten sie gebraucht werden.
Zudem hat man mit der Software einen guten Überblick über die vorhandenen Guter im Unternehmen.
\newpage

\section{Funktionale Anforderungen}

\subsection{Muss-Ziele}
\subsubsection{An- und Abmeldung}
\textbf{/LF0010/} Login
\par
\begingroup
\leftskip=1cm
\noindent Der Anwender der App soll sich mit seinem Usernamen, sowie mit seinem Passwort, anmelden können.\\
\par
\endgroup

\leftskip=-1.5em
\textbf{/LF0020/} Logout
\par
\begingroup
\leftskip=1cm
\noindent Der Anwender soll sich jederzeit ausloggen können.\\
\par
\endgroup

\textbf{/LF0030/} Registrieren neuer Anwender
\par
\begingroup
\leftskip=1cm
\noindent Es soll die Möglichkeit bestehen neue Nutzer der App hinzuzufügen und ihnen ihre Daten zur Anmeldung zu übermitteln.\\
\par
\endgroup

\leftskip=0em
\subsubsection{Datenzugriff}
\textbf{/LF0110/} Anzeigen aller erfassten Items
\par
\begingroup
\leftskip=1cm
\noindent Alle erfassten Items sollen als Tabelle ausgegeben werden.
\begin{enumerate}
\leftskip=3em
\item[a)] Benutzerdefinierte Spalten
\item[] Die anzuzeigenden Spalten können von jedem Anwender individuell definiert werden.
\end{enumerate}
\par
\endgroup

\leftskip=-1.5em
\textbf{/LF0120/} Suche
\par
\begingroup
\leftskip=1cm
\noindent Es soll ein Suchfeld vorhanden sein, womit es einem Anwender gestattet ist gezielt nach Items zu suchen.
\begin{enumerate}
\leftskip=3em
\item[a)] Typspezifische Suche
\item[] Es soll nach einem spezifischen Typ gesucht werden können, wie zum Beispiel nach dem Typ $"$Laptop$"$.
\item[b)]Feldspezifische Suche
\item[] Es soll die Suche nach bestimmten Werten in Feldern ermöglicht sein, zum Beispiel die Suche nach "Benutzern$"$ mit dem Wert $"$Max Mustermann$"$.
\end{enumerate}
\par
\endgroup

\leftskip=0em
\subsubsection{Dateneingabe}
\textbf{/LF0210/} Neue Items hinzufügen
\par
\begingroup
\leftskip=1cm
\noindent Der Anwender soll neue Items hinzufügen können, mit den, passend zum Itemtyp, entsprechenden Eigenschaften.\\
\par
\endgroup

\leftskip=0em

\leftskip=-1.5em
\textbf{/LF0220/} Anhang von Dokumenten
\par
\begingroup
\leftskip=1cm
\noindent Es soll möglich sein einem Item Dateien (PDF, PNG etc.) versioniert anhängen zu können.\\
\par
\endgroup

\leftskip=-1.5em
\textbf{/LF0230/} Bearbeiten von Items
\par
\begingroup
\leftskip=1cm
\noindent Der Anwender soll die Möglichkeit haben bereits vorhandene Items überarbeiten zu können.\\
\par
\endgroup

\leftskip=-1.5em
\textbf{/LF0240/} Entfernen von Items
\par
\begingroup
\leftskip=1cm
\noindent Es soll für den Anwender möglich sein, Items restlos löschen zu können.\\
\par
\endgroup

\leftskip=0em
\subsubsection{Typeneingabe}

\textbf{/LF0310/} Hinzufügen neuer Itemtypen
\par
\begingroup
\leftskip=1cm
\noindent Es soll möglich sein neue Typen von Items hinzufügen zu können.\\
\par
\endgroup

\leftskip=-1.5em
\textbf{/LF0320/} Bearbeitung von Itemtypen
\par
\begingroup
\leftskip=1cm
\noindent Der Anwender soll Typen überarbeiten können, zu dem sollen auch alle Items, welche dem Typ zugehörig sind, aktualisiert werden.\\
\par
\endgroup

\leftskip=-1.5em
\textbf{/LF0330/} Löschen von Itemtypen
\par
\begingroup
\leftskip=1cm
\noindent Das Löschen von bestimmten Itemtypen soll implementiert sein.\\
\par
\endgroup

\leftskip=0em
\subsubsection{Administration}

\textbf{/LF0410/} Firmenverwaltung
\par
\begingroup
\leftskip=1cm
\noindent Es soll möglich sein neue Unternehmen/Firmen der App hinzuzufügen und zu bearbeiten, sowie mehrere Unternehmen/Firmen unter einer Dachgesellschaft zusammenzufügen.\\
\par
\endgroup

\leftskip=-1.5em
\textbf{/LF0420/} Typenverwaltung
\par
\begingroup
\leftskip=1cm
\noindent Es soll das Hinzufügen, Bearbeiten und Löschen von Typen möglich sein. (siehe Typeneingabe)\\
\par
\endgroup

\leftskip=-1.5em
\textbf{/LF0430/} Globale Pflichtfelder
\par
\begingroup
\leftskip=1cm
\noindent Es soll das Hinzufügen und Bearbeiten von Feldern ermöglicht sein, welche in jedem Item vorkommen.\\
\par
\endgroup

\leftskip=0em
\subsection{Kann-Ziele}

\textbf{/LF0510/} E-Mail Adresse ändern
\par
\begingroup
\leftskip=1cm
\noindent Für den Anwender soll die Option bestehen, seine hinterlegte E-Mail Adresse jederzeit ändern zu können.\\
\par
\endgroup

\leftskip=-1.5em
\textbf{/LF0520/} Passwort ändern
\par
\begingroup
\leftskip=1cm
\noindent Der Anwender soll sein Passwort ändern können, solange wie er eingeloggt ist.\\
\par
\endgroup

\leftskip=-1.5em
\textbf{/LF0520/} Mehrsprachigkeit
\par
\begingroup
\leftskip=1cm
\noindent Die Anwendung soll in mehreren Sprachen verfügbar sein und übersetzungen sollen einfach hinzugefügt werden.\\
\par
\endgroup

\leftskip=-1.5em
\textbf{/LF0520/} Itemtyp Vorlagen
\par
\begingroup
\leftskip=1cm
\noindent Typische Itemtypen, welche oft in Unternehmen vorhanden sind, 
sollen als Vorlage bereits zur Verfügung stehen um das Aufsetzen der Software zu vereinfachen.\\
\par
\endgroup

\textbf{/LF0530/} Wiederherstellung des Passworts
\par
\begingroup
\leftskip=1cm
\noindent Für den Fall das ein Anwender sein Passwort vergisst, soll es die Option geben ein neues zu erhalten.\\
\par
\endgroup

\textbf{/LF0540/} Sortieren von Items
\par
\begingroup
\leftskip=1cm
\noindent Die Sortierung von Items nach dem Nutzerwunsch soll möglich sein.\\
\par
\endgroup

\textbf{/LF0550/} Filtern von Items
\par
\begingroup
\leftskip=1cm
\noindent Der Anwender soll, nach seinen Wünschen, die vorhandene Liste filtern können, zum Beispiel damit nur Items mit dem Feld $"$defekt:false$"$ angezeigt werden.\\
\par
\endgroup

\textbf{/LF0560/} Item Feld $"$Input$"$
\par
\begingroup
\leftskip=1cm
\noindent Das Feld $"$Input$"$, soll die Eingabe erleichtern, so könnte es zum Beispiel Color-Picker für Farben, Fileuploads etc. zur Verfügung stellen.\\
\par
\endgroup

\newpage
\textbf{/LF0570/} Verlinkung von Items
\par
\begingroup
\leftskip=1cm
\noindent Es soll möglich sein Items untereinander zu verlinken, zum Beispiel existiert ein Item $"$Location X$"$ , welches ein Feld von Item $"$Y$"$ ist.
Dies kann auf Orte, Personen etc. übertragen werden.\\
\par
\endgroup

\textbf{/LF0580/} Pflichfelder von Itemtypen
\par
\begingroup
\leftskip=1cm
\noindent Itemtypen sollen, falls angebracht, Pflichtfelder zugeschrieben bekommen, welche bei jedem neuem Item des Typs vorhanden sein müssen.\\
\par
\endgroup

\textbf{/LF0590/} Rollenverwaltung
\par
\begingroup
\leftskip=1cm
\noindent Es soll, vorhandenen Anwenderrollen, Berechtigungen hinzugefügt, bzw. entfernt werden können.\\
\par
\endgroup

\vskip 10em
\leftskip=0em
\section{Nicht-funktionale Anforderungen}
\subsection{Muss-Ziele}
\subsubsection{Datenschutz und Anonymität}

\textbf{/LN0010/} Löschen von Profilen
\par
\begingroup
\leftskip=1cm
\noindent Der Admin soll, genau wie bei der Bearbeitung von Profildaten, stets die Möglichkeit haben Profile zu löschen, um Personen, welche nicht mehr zum Unternehmen gehören, jederzeit entfernen zu können.\\
\par
\endgroup

\leftskip=-1.5em
\textbf{/LN0020/} Sicherheit der Anmeldedaten
\par
\begingroup
\leftskip=1cm
\noindent Um die Sicherheit der Anmeldedaten zu gewährleisten, sollen diese verschlüsselt gespeichert werden.\\
\par
\endgroup

\leftskip=0em
\subsubsection{Fehlerunanfälligkeit}

\textbf{/LN0110/} Validieren von Benutzereingaben
\par
\begingroup
\leftskip=1cm
\noindent Sämtliche Benutzereingaben müssen validiert werden, sollte die Validierung fehlschlagen, darf die Eingabe nicht übernommen werden.\\
\par
\endgroup

\leftskip=-1.5em
\textbf{/LN0120/} \textbf{Datensicherheit}
\par
\begingroup
\leftskip=1cm
\noindent Sollte es zu internen Fehlern kommen, ist der Schaden (Datenverlust etc.) so gering wie möglich zu halten.
Ein Absturz der App sollte unter allen Umständen vermieden werden.\\
\par
\endgroup

\leftskip=0em
\subsubsection{Wartung}

\textbf{/LN0210/} Dokumentationen des Programmcodes
\par
\begingroup
\leftskip=1cm
\noindent Der Programmcode soll vollständig kommentiert und dokumentiert sein, um eine spätere Wartung zu erleichtern.\\
\par
\endgroup

\newpage
\leftskip=-1.5em
\textbf{/LN0220/} Modulierbarkeit und Wartung
\par
\begingroup
\leftskip=1cm
\noindent Die Struktur des Programmcodes soll verhältnismäßig 'leicht' aufgebaut sein, so dass, von allen Personen der eventuellen Weiterentwicklung, Änderungen gezielt vorgenommen werden können.\\
\par
\endgroup

\leftskip=0em
\subsubsection{Benutzeroberfläche}

\textbf{/LN0310/} Intuitive grafische Interaktion
\par
\begingroup
\leftskip=1cm
\noindent Das UI (User-Interface) soll so einfach wie möglich gehalten werden, damit es intuitiv von dem Benutzer verwendet werden kann.\\
\par
\endgroup

\leftskip=0em
\subsection{Kann-Ziele}
\textbf{/LN0410/} Browsersupport
\par
\begingroup
\leftskip=1cm
\noindent Das Programm soll von so vielen Browsern wie möglich unterstützt werden.\\
\par
\endgroup

\leftskip=-1.5em
\textbf{/LN0420/} Effiziente Implementierung
\par
\begingroup
\leftskip=1cm
\noindent Das Programm soll möglichst effizient laufen, das heißt eine möglichst kleine Komplexität in Laufzeit und Speicherbedarf.\\
\par
\endgroup

\newpage
\leftskip=0em
\section{Qualitätsmatrix nach ISO 25010}
\begin{figure}[h]
\begin{center}
\includegraphics[scale=1.0]{Qualitaet.png}
\end{center}
\end{figure}
Besonders hohe Priorität hat bei uns die Sicherheit.
Unternehmenswerte und Dokumente sollen ausschließlich von autorisierten Personen gesehen und bearbeitet werden können, da es sich hierbei um sehr sensible Daten handeln kann. Ebenfalls wichtig ist die Benutzbarkeit, da jeder Mitarbeiter die App nutzen soll und deshalb Schulungen für die Benutzung nicht notwendig sein sollen.
Effizienz hingegen wird von uns als eher unwichtig eingestuft, da die Mengen an Daten überschaubar bleiben und keine rechenintensiven Berechnungen Teil der App sind.
Die Kompatibilität haben wir ebenfalls niedrig eingestuft, da wir vorerst nur den Chrome-Browser unterstützen möchten.

\vskip 10em
\section{Lieferumfang und Abnahmekriterien}

\subsection{Lieferumfang}
Geliefert werden soll eine Webanwendung zur Verwaltung unternehmensspezifischer Werte, welche wesentliche Funktionalitäten bereitstellt.
Sie soll speziell für Bürounternehmen konzipiert sein.
Es soll die Benutzeroberfläche, sowie die Nutzerverwaltung, bereits implementiert sein. Das Back-End wird durch $"$mongoDB$"$ realisiert, um alle nötigen Informationen speichern, verwalten und synchronisieren zu können.
Des Weiteren wird eine vollständige Dokumentation zur Verfügung gestellt.

\subsection{Abnahmekriterien}
Als Hauptabnahmekriterium wird ein vollständig funktionstüchtiger Prototyp des Inventarisierungstools festgelegt.
Es sollen alle Muss-Ziele und so viele Kann-Ziele wie möglich fertiggestellt sein.
Der Lieferumfang soll eingehalten werden.
Neben der vollständigen Dokumentation das Projekts, sollte darauf geachtet werden, dass die Qualitätsmatrix eingehalten wird.

\vskip 10em
\section{Vorprojekt}
Im Rahmen des Vorprojekts soll das wesentliche Grundgerüst, bestehend aus Datenbank und Server- und Client-Grundkonzept, erstellt werden.
Zu diesem Zeitpunkt sollte das Hinzufügen und Löschen von Items, sowie deren Darstellung, möglich sein.
Die Darstellung der Items sollte am besten in Form einer Liste vorgenommen werden.
Es muss ein Grundkonzept für diese Items bereits konzipiert und implementiert sein. Dachgesellschaften, sowie die Darstellung und Modifizierung der Items von den dazugehörigen Unternehmen, sollen ebenfalls fertiggestellt sein.
Die Rechtevergabe an einzelne Personengruppen, damit Diese nicht alles einsehen können, sondern nur das was vom Admin für sie freigegeben ist, soll auch funktionstüchtig sein.

\vskip 10em
\section{Glossar}
\textbf{Inventar}
\par
\begingroup
\leftskip=1cm
\noindent ist die Gesamtheit der zu einem Betrieb/Unternehmen gehörenden
Einrichtungsgegenstände und Vermögenswerten.
Zudem muss das Inventar jährlich inventarisiert werden.\\
\par
\endgroup

\leftskip=-1.5em
\textbf{Inventarisierung}
\par
\begingroup
\leftskip=1cm
\noindent ist die Bestandsaufnahme von unternehmensspezifischen Objekten
in Hinsicht auf spezielle Merkmale selbiger.
Sie bildet die Informationsgrundlage für
Kosten- und Leistungsrechnungen sowie zur Ermittlung von einem potentiellen Reinvestitionsbedarf.\\
\par
\endgroup

\textbf{Item}
\par
\begingroup
\leftskip=1cm
\noindent ist die Bezeichnung der konkreten, unternehmensspezifischen Objekten welches einem Unternehmen gehört.\\
\par
\endgroup

\textbf{Feld}
\par
\begingroup
\leftskip=1cm
\noindent ist der Begriff für die Merkmale eines Items und beinhaltet Metainformationen, 
wie z.B. ob es ein Pflichfeld ist und von welchem Datentyp es ist.\\
\par
\endgroup

\textbf{Itemtyp}
\par
\begingroup
\leftskip=1cm
\noindent ist das Grundgerüst, welches ein Item beschreibt. 
Der Itemtyp beinhaltet alle Felder, welche ein Item haben kann.\\
\par
\endgroup

\textbf{Dachgesellschaft}
\par
\begingroup
\leftskip=1cm
\noindent beschreibt ein Wirtschaftskonzept, bei dem die Dachgesellschaft (in Literatur auch Holdinggesellschaft) als Verwaltung der zugehörigen Unternehmen dient.\\
\par
\endgroup

\end{document}
