\documentclass{beamer}

\mode<presentation>

\usetheme[width=2.2cm]{Berkeley}

\usepackage[utf8]{inputenc}
\usepackage[ngerman]{babel}
\usepackage{graphicx}

\title{Präsentation zum Lastenheft}
\author{Gruppe: ak18b}
\date{06. Dezember 2018}

\begin{document}

\setbeamertemplate{sidebar left}{}
\begin{frame}
\begin{figure}
\includegraphics[scale=0.25]{Logo.png}
\end{figure}
\begin{center}
\vskip -1em
\tiny Wintersemester 2018/19\\
\tiny Softwaretechnikpraktikum
\end{center}
\vskip -2em
\titlepage
\end{frame}

\begin{frame}
\frametitle{Inhaltsverzeichnis}
\small
\tableofcontents
\end{frame}

\section{Funktionale Anforderungen}

\subsection{Muss-Ziele}

\subsubsection{An- und Abmeldung}

\setbeamertemplate{sidebar left}[sidebar theme]
\makeatletter
  \setbeamertemplate{sidebar \beamer@sidebarside}%{sidebar theme}
  {
    \beamer@tempdim=\beamer@sidebarwidth%
    \advance\beamer@tempdim by -6pt%
    \insertverticalnavigation{\beamer@sidebarwidth}%
    \vfill
    \ifx\beamer@sidebarside\beamer@lefttext%
    \else%
      \usebeamercolor{normal text}%
      \llap{\usebeamertemplate***{navigation symbols}\hskip0.1cm}%
      \vskip2pt%
    \fi%
  }%
\makeatother
\begin{frame}
\frametitle{An- und Abmeldung}

\begin{block}{Login}
\begin{itemize}
\item über Username und Passwort
\end{itemize}
\end{block}

\begin{block}{Logout}
\begin{itemize}
\item jederzeit möglich
\end{itemize}
\end{block}

\begin{block}{Registrierung neuer Anwender}
\begin{itemize}
\item jederzeit neue Nutzer hinzufügen
\item neue Nutzer Logindaten übermitteln
\end{itemize}
\end{block}

\end{frame}

\subsubsection{Datenzugriff}

\begin{frame}
\frametitle{Datenzugriff}

\begin{block}{Anzeigen aller erfassten Items}
\begin{itemize}
\item erfasste Items in einer Tabelle ausgeben
\item anzuzeigende Spalten von jedem Anwender definierbar
\end{itemize}
\end{block}

\begin{block}{Suche}
\begin{itemize}
\item gezieltes suchen nach Items über Suchfeld
\begin{itemize}
\item suche nach Typen (z.B. $"$Laptop$"$)
\item suche nach Feldern (z.B. $"$Benutzer$"$)
\end{itemize}
\end{itemize}
\end{block}

\end{frame}

\subsubsection{Dateneingabe}

\begin{frame}
\frametitle{Dateneingabe}

\begin{block}{Neue Items hinzufügen}
\begin{itemize}
\item Items hinzufügen, mit entsprechenden Eigenschaften
\end{itemize}
\end{block}

\begin{block}{Anhang von Dokumenten}
\begin{itemize}
\item Items versionierte Dateien anhängen
\end{itemize}
\end{block}

\begin{block}{Bearbeiten von Items}
\begin{itemize}
\item vorhandene Items überarbeiten/verändern
\end{itemize}
\end{block}

\begin{block}{Entfernen von Items}
\begin{itemize}
\item restloses löschen von Items
\end{itemize}
\end{block}

\end{frame}

\subsubsection{Typeneingabe}

\begin{frame}
\frametitle{Typeneingabe}

\begin{block}{Hinzufügen von neuen Itemtypen}
\begin{itemize}
\item neue Itemtypen hinzufügen
\end{itemize}
\end{block}

\begin{block}{Bearbeiten von Itemtypen}
\begin{itemize}
\item Typen überarbeiten/verändern
\begin{itemize}
\item zugehörige Items aktualisieren
\end{itemize}
\end{itemize}
\end{block}

\begin{block}{Löschen von Itemtypen}
\begin{itemize}
\item löschen bestimmter Itemtypen
\end{itemize}
\end{block}

\end{frame}

\subsubsection{Administration}

\begin{frame}
\frametitle{Administration}

\begin{block}{Firmenverwaltung}
\begin{itemize}
\item Unternehmen/Firma der App hinzufügen/bearbeiten
\item mehrere Unternehmen/Firmen unter Dachgesellschaft zusammenfassen
\end{itemize}
\end{block}

\begin{block}{Typenverwaltung}
\begin{itemize}
\item hinzufügen, bearbeiten, löschen von Itemtypen
\end{itemize}
\end{block}

\begin{block}{Globale Pflichtfelder}
\begin{itemize}
\item hinzufügen/bearbeiten von globalen Pflichtfeldern
\end{itemize}
\end{block}

\end{frame}

\subsection{Kann-Ziele}

\begin{frame}
\frametitle{Kann-Ziele}

\begin{block}{E-Mail Adresse ändern}
\begin{itemize}
\item hinterlegte E-Mail Adresse jederzeit veränderbar
\end{itemize}
\end{block}

\begin{block}{Passwort ändern}
\begin{itemize}
\item Passwort ändern möglich
\end{itemize}
\end{block}

\begin{block}{Mehrsprachigkeit}
\begin{itemize}
\item mehrere Sprachen verfügbar
\item Übersetzungen einfach hinzuzufügen
\end{itemize}
\end{block}
\end{frame}

\begin{frame}
\frametitle{Kann-Ziele}

\begin{block}{Itemtyp Vorlagen}
\begin{itemize}
\item häufig genutzte Itemtypen, bereits hinterlegt
\end{itemize}
\end{block}

\begin{block}{Wiederherstellung des Passworts}
\begin{itemize}
\item Option neues Passwort zu erhalten
\end{itemize}
\end{block}

\begin{block}{Sortieren von Items}
\begin{itemize}
\item nach Nutzerwunsch möglich
\end{itemize}
\end{block}

\end{frame}

\begin{frame}
\frametitle{Kann-Ziele}
\begin{block}{Filtern von Items}
\begin{itemize}
\item filtern nach Nutzerwunsch (z.B. nur $"$defekt=false$"$)
\end{itemize}
\end{block}

\begin{block}{Itemfeld $"$Input$"$}
\begin{itemize}
\item Erleichterung der Eingabe
\item Beispiele: Colorpicker, Fileuploads etc.
\end{itemize}
\end{block}

\begin{block}{Verlinkung von Items}
\begin{itemize}
\item Items untereinander verlinken
\end{itemize}
\end{block}

\end{frame}

\begin{frame}
\frametitle{Kann-Ziele}

\begin{block}{Pflichtfelder von Itemtypen}
\begin{itemize}
\item Itemtypen mit Pflichtfeldern versehen (falls angebracht)
\item neue Item des Typs erhält diese Felder
\end{itemize}
\end{block}

\begin{block}{Rollenverwaltung}
\begin{itemize}
\item Anwenderrollen, Berechtigungen hinzufügen/bearbeiten/entfernen
\end{itemize}
\end{block}

\end{frame}

\section{Nicht-Funktionale Anforderungen}

\subsection{Muss-Ziele}

\subsubsection{Datenschutz und Anonymität}

\begin{frame}
\frametitle{Datenschutz und Anonymität}

\begin{block}{Löschen von Profilen}
\begin{itemize}
\item Admin hat Berechtigung Benutzer zu löschen
\end{itemize}
\end{block}

\begin{block}{Sicherheit der Anmeldedaten}
\begin{itemize}
\item Verschlüsselung von Nutzerdaten
\item verschlüsseln durch hashen
\end{itemize}
\end{block}

\end{frame}

\subsubsection{Fehlerunanfälligkeit}

\begin{frame}
\frametitle{Fehlerunanfälligkeit}

\begin{block}{Validieren von Benutzereingaben}
\begin{itemize}
\item alle Eingaben validieren
\item bei Fehlschlag Änderung nicht übernehmen
\end{itemize}
\end{block}

\begin{block}{Datensicherheit}
\begin{itemize}
\item bei internen Fehlern Schaden gering wie möglich halten
\item Absturz vermeiden
\end{itemize}
\end{block}

\end{frame}

\subsubsection{Wartung}

\begin{frame}
\frametitle{Wartung}

\begin{block}{Dokumentation des Programmcodes}
\begin{itemize}
\item Programmcode vollständig kommentiert/dokumentiert
\end{itemize}
\end{block}

\begin{block}{Modulierbarkeit und Wartung}
\begin{itemize}
\item leichte Programmstruktur
\end{itemize}
\end{block}

\end{frame}

\subsubsection{Benutzeroberfläche}

\begin{frame}
\frametitle{Benutzeroberfläche}

\begin{block}{Intuitive grafische Interaktion}
\begin{itemize}
\item UI so intuitiv wie möglich halten
\item einfache Bedienung garantieren
\end{itemize}
\end{block}

\end{frame}

\subsection{Kann-Ziele}

\begin{frame}
\frametitle{Kann-Ziele}

\begin{block}{Browsersupport}
\begin{itemize}
\item soll viele Browser unterstützen
\end{itemize}
\end{block}

\begin{block}{Effiziente Implementierung}
\begin{itemize}
\item kleine Komplexität in Laufzeit und Speicherbedarf
\end{itemize}
\end{block}

\end{frame}

\section{Qualitätsmatrix nach ISO 25010}

\begin{frame}
\frametitle{Qualitätsmatrix nach ISO 25010}

\begin{figure}
\vskip -1em
\includegraphics[scale=0.35]{Qualitaet.png}
\vskip -1.3em
\end{figure}

\begin{block}{Beschreibung}
\begin{itemize}
\item hohe Priorität: Funktionalität, Sicherheit, Benutzbarkeit
\item mittlere Priorität: Zuverlässigkeit, Wartbarkeit, Portierbarkeit
\item niedrige Priorität: Effizienz, Kompatibilität
\end{itemize}
\end{block}

\end{frame}

\section{Lieferumfang}

\begin{frame}
\frametitle{Lieferumfang}
\begin{block}{}
\begin{itemize}
\item Webanwendung zur Verwaltung unternehmensspezifischer Werte
\item speziell für Bürounternehmen konzipiert
\item Benutzeroberfläche/Nutzerverwaltung implementiert
\item Back-End mit $"$mongoDB$"$
\item vollständige Dokumentation
\end{itemize}
\end{block}
\end{frame}

\section{Abnahmekriterien}

\begin{frame}
\frametitle{Abnahmekriterien}
\begin{block}{}
\begin{itemize}
\item funktionstüchtiger Prototyp
\item alle Muss-Ziele fertiggestellt
\item so viele Kann-Ziele wie möglich fertiggestellt
\item Lieferumfang einhalten
\item Qualitätsmatrix einhalten
\end{itemize}
\end{block}
\end{frame}

\section{Vorprojekt}

\begin{frame}
\frametitle{Vorprojekt}

\begin{block}{}
\begin{itemize}
\item Grundgerüst erstellen
\begin{itemize}
\item Datenbank
\item Server- und Client-Grundkonzept
\end{itemize}
\item Hinzufügen/Löschen und Darstellung von Items
\begin{itemize}
\item Darstellung als Tabelle
\end{itemize}
\item Grundkonzept für Items konzipiert und implementiert
\item Dachgesellschaft implementiert
\begin{itemize}
\item Darstellung und Modifizierung von Items zgh. zu entsprechenden Unternehmen
\end{itemize}
\item Rechtevergabe an Nutzer vom Admin
\end{itemize}
\end{block}
\end{frame}

\end{document}
